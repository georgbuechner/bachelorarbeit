\documentclass[12pt, a4paper, openany]{report}
\usepackage[left=3cm,top=3cm, bottom=3cm, right=4cm]{geometry}

% my custom stlye and functions stuff
\usepackage{mystyle}
\usepackage{csquotes}

\begin{document}
\begin{itemize}
    \item[] \textbf{b. Endlich und unendlich} 
    \item[1.] Wiederholung/ Zusammenfassung der 2. Natur
    \item[1.1] Zweite Natur ist ein Begriff der Kritik, er zeigt auf, inwieweit der Geist als zweite Natur im \ul{Widerspruch zu seinem eigenen Begriff} steht. 
    Gutes Zitat zum Anknüpfen: \qq{Tod des Geistes durch den Geist und im Geist} (\cite[][43]{menke_autonomie_2018})
    \item[1.2.] Zweite Natur ist allerdings auch darum ein kritischer Begriff, weil er die Entstehung der zweite Natur beschreibt.
    Und zwar beschreibt er sie so, dass es eine \ul{Entstehung aus dem Geist selbst heraus} ist. 
    Der Begriff ist also kritisch gegen den Geist selber, nicht nur gegen die kritische Form, die dieser annimmt
    \item[] \textit{Etwas Anthropologisches}
    \item[2.] Das \qq{Problem} der zweiten Natur ist ein eigentlich anthropologisches Problem: 
    \qq{Notwendigkeit und Mangel der zweiten Natur im Geist sind \ul{relativ auf die Notwendigkeit (und den Mangel) seiner endlichen, menschlichen Gestalt}} (\cite[][138]{menke_autonomie_2018})
    \item[2.1] \qq{Zweite Natur steht \ul{\emph{zwischen} der Theorie} des endlichen und des absoluten Geistes.} (\cite[][140]{menke_autonomie_2018}).
    Der \emph{endliche} Geist unterliegt der (seiner) Selbstverkehrung in zweite Natur.
    Der \emph{absolute} Geist hingegen \qq{ist der Geist, dessen \emph{eigenes} Gesetz die Selbstverkehrung ist} (\cite[][140]{menke_autonomie_2018}). 
    Er vollzieht in Freiheit seine Selbstverkehrung und zweite Natur.
    \qq{Es ist daher das \ul{eigene Bewegungsgesetz} des absoluten Geistes, in den endlichen Geist zurückzufallen} (\cite[][140]{menke_autonomie_2018}).
    \item[2.1.1] Der endliche Geist ist \ul{durchgehend von der zweiten Natur beherrscht}, er kann niemals aufhören, naturverfallener Geist zu sein. (\cite[Vgl.][139]{menke_autonomie_2018})
    \item[2.1.2] Der Begriff der zweiten Natur schließt die Perspektive auf ihre Überwindung, und auf den \emph{absoluten} Geist mit ein.
    Im Zurückführen der zweiten Natur auf ihren \qq{Geburtsakt} \emph{im} Geist, macht Hegel dadurch den Geist selbst für seine Selbstverkehrung verantwortlich.
    Er ist der Grund für seine Verfehlung.
    Indem er aber Grund für seine eigene Verfehlung ist und diese hervorbringt, \qq{ist er in diesem Hervorbringen mehr und anderes, als endlicher Geist} (\cite[][140]{menke_autonomie_2018}) - er \emph{unterliegt} ihr nicht nur, sondern bringt sie hervor.
    \qq{In der Perspektive einer genealogischen Kritik der zweiten Natur ist der Geist daher immer schon \ul{mehr als der endliche Geist}, in den er sich verkehrt.     
    Dieses Mehr - das als \ul{Grund} der Verkehrung ins Endliche \ul{kein Jenseits} sein kann - heißt >>absoluter Geist<<.} (\cite[][140]{menke_autonomie_2018}
    \item[] \textit{Die Manifestation des Geistes}
    \item[3.] zweite Natur = Manifestation des Geistes. 
    \qq{Das \emph{Wesen} des Geistes ist [...] die \emph{Freiheit}} (\cite[][§ 382, S. 25]{hegel_enzyklopädie_1969}.
    Zugleich ist (wie wir auch im ersten Abschnitt gesehen haben) Freiheit notwendig auch Bestimmtheit.
    Zu dieser Bestimmtheit kommt der gelangt der Geist durch seine Manifestation: \qq{seine Bestimmtheit ist [sein] Offenbaren selbst.} (\cite[][141]{menke_autonomie_2018}
    Hegel: \qq{Das \emph{Offenbaren} [...] ist als Offenbaren des Geistes, der frei ist, \emph{Setzen} der Natur als \emph{seiner} Welt; ein Setzen, das als Reflexion zugleich \emph{Voraussetzen} der Welt als selbstständiger Natur ist.} (\cite[][§ 384, S. 29])
    Hier sind zwei Seiten kurz hervorzuheben: 1. das die Bestimmtheit hier in der Form des Bei-Sich-Selbst-Sein-Im-Anderen auftritt, und dass sich hier zum ersten mal die Idee offenbart und als abstrakte Idee übergeht in die äußere Natur. (\cite[Vgl.][141 (Fußnote)]{menke_autonomie_2018})\\
    Die hat die beiden Seiten:
    \item[3.1.1.] Das Setzen seiner Welt und sich diese als seine anzueignen.
    \qq{[E]ine >>Welt<< zu setzen, die >>seine<< eigene ist.} (\cite[][141]{menke_autonomie_2018}) 
    -> \emph{Aneignen - Setzen der Natur als Welt.}
    \item[3.1.2.] Diese selbst hervorgebrachte Welt sich wieder als \qq{selbstständige Natur gegenüberzustellen} (\cite[][141]{menke_autonomie_2018}).
    Menke benutzt hier auch \emph{Voraussetzen}.
    -> \emph{Verselbstständigung - Voraussetzen der Welt als Natur}
    \item[3.2] Im Gegensatz zu Hegel meint Menke, diese zwei Bereiche seien nicht der Natur und dem Geist klar zugeordnet, sondern diese Operationen (Setzen und Voraussetzen), schaffen erst die Differenz zwischen Geist und Natur.
    Was wir zunächst als Offenbarung, oder Manifestation beschrieben haben, nennt Menke hier \qq{Selbstverwirklichung des Geistes}, und sie sei immer Setzen und Voraussetzen zugleich.\\
    Diese Doppelstruktur ist wie folgt zu verstehen:
    \item[3.2.1 a)] \qq{Das >>Vorraussetzen einer Welt als selbstständiger Natur<< muss zugleich ein Setzen sein, weil es sonst nicht das Voraussetzen einer \emph{Welt} ist.} (\cite[][142]{menke_autonomie_2018})
    (Welt ist hier also als etwas genuin geistiges vorgestellt, was gesetzt sein muss.
    Damit die Natur \emph{als Welt} gesetzt wird, muss sie also auch gesetzt sein) 
    \item[3.2.1 b)] \qq{[D]as >>Setzen der Natur als seiner Welt<< [muß] zugleich ein Voraussetzen sein, weil es sonst nicht ein Setzen von \emph{Natur} ist} (\cite[][142]{menke_autonomie_2018}).
    (Das was gesetzt wird, wir als selbstständig gegenüber dem Setzen gesetzt und dadurch \emph{voraus} gesetzt.)
    Es wird also so gesetzt, als sei es nichtgesetzt, als sei es \emph{Sein}.)
    \item[3.3] In dieser Doppelbestimmung \qq{definiert den begrifflichen Ort der zweiten Natur in der Selbstverwirklichung des Geistes} (\cite[][142]{menke_autonomie_2018}).  
    Dieser Ort ist (erneut) doppelt bestimmbar:
    \item[3.3.1 Affirmation] Das Offenbaren des Giestes, das Setzen von Sein, ist \qq{die \emph{Affirmation} und \emph{Wahrheit} seiner Freiheit} (\cite[][§384, S. 29]{hegel_enzyklopädie_1969}).
    Der Begriff der zweiten Natur ist deshalb affirmativ, weil er in seiner Form (Sein zu setzen) die From des absoluten Geists enthält, oder ist: 
    \qq{[D]as vom Geist gesetzte [muß] zugleich als ein unmittelbar Seieindes gefaßt werden. 
    Dies geschieht [...] auf dem Standpunkt des \emph{absoluten} Geistes} (\cite[][§ 385 Z, S. 34]{hegel_enzyklopädie_1969}).
    \qq{Der sbsolute Geist \emph{ist} zweite Natur} (\cite[][144]{menke_autonomie_2018}).
    \item[3.3.1 Kritik] Setzten und Sein fallen im endlichen Sein auseinander.
    Im endliche Geist erscheint zweite Natur, stets als Setzung \emph{oder} Sein, während sie ihrem Begriff nach Setzung \emph{des} Seins ist. 
    Der endliche Geist erschient in diesem Gegensatz, um so erscheinen zu können, muss er ihn aber selbst hervorbringen können, er ist also zugleich \qq{Sein \emph{gegen} Setzung [...] und Sein \emph{durch} Setzung} (\cite[][144]{menke_autonomie_2018}).
    (Hier kann man einen kleinen Bogen schlagen zu Nietzsche, nach dem die zweite Natur, der Schein, eben genau so erscheinen soll, dass er diese Einheit von Setzen und Sein ist.
    Nietzsche verlangt, dass Schein für ihn 1. Natur selbst ist und 2. aber merken lässt, dass sie nur Geist ist.)
    \item[3.3.2] Endlicher Geist zu sein, beduetet also absoluter Geist gewesen zu sein (sonst wäre das Setzen von Sein nicht möglich). 
    Aber absoluter Geist zu sein bedeutet, endlicher Geist zu werden, denn das gesetzte Sein erscheint nicht als Einheit von Setzt udn Sein, sondern stets als dessen Gegensatz (im endlichen Geist).
    \item[3.4 Meine Kritik] Letztendlich liegt de kritische Begriff aber allein darin, dass die zweite Natur niemals als diese Einheit von Setzen und Sein, sondern sets als deren Gegensatz erscheinen. 
    Warum kann nicht dieses Erscheinen sich verändern?
    Ist das Argument Hegels, bzw. Menkes ein rein anthropologisches Argument?
    Der Mensch ist nunmal endlicher Geist? 
    Warum muss das Setzen in Sein umschlagen, auf eine Art und Weise, dass das Gesetzt-Sein nicht darin aufgehoben und noch zu sehen ist?
    Menke beschreibt zwar, wie es in dem Begriff der zwieten Natur eingeschrieben ist, dass sie nur Setzen von selbstständigen Sein sein kann, wenn sich das Sein auch gegen das Setzen verselbstständigt, aber im absoluten Geist scheint diese Einheit ja zu bestehen, weshalb muss sich dem endlichen Geist notwendig abgehen?
    \item[] \textbf{c. Kritik und Affirmation}
    \item [1.] 
\end{itemize}
\begin{itemize}
    \item[] \textbf{c. Kritik und Affirmation}
    \item[1.] Neubestimmung der Struktur der zweiten Natur, auf Grund des \qq{unauflösbaren Ineinander von endlichem und absolutem Geist} (\cite[][145]{menke_autonomie_2018}).
    \item[1.1] Im geistigen Mechanismus hatten wir die STruktur der zweiten Natur so verstanden, dass sie geistig ist, so fern sie die freie Aneignung von Fähigkeiten in der Bildung ist, also ein Wollen \todo{Inwiefern ist das Wollen möglich. \cite[Vgl][28ff.]{menke_autonomie_2018}}, oder Setzen und 2. unfrei oder mechanisch, \qq{weil sie, einmal gesetzt, selbstständig und unbewußt wirkend ist.} (\cite[][145]{menke_autonomie_2018})
    \item[1.2.] Im vorherigen Abschnitt hatten wir jedoch die mechanische oder unfreie Seite der zweiten Natur in sich selbst als doppeldeutig beschrieben: 
    \qq{Sie beschreibt zugleich - affirmativ - die Verwirklichung \emph{und} - kritisch- den Fall des Geistes} (\cite[][145]{menke_autonomie_2018})!!
    Der Freiheitsverlust ist zugleich die Selbstüberschreitung des endlichen Geistes zum absoluten. (Beispiel Kunstwerk)
    \item[2.] -> Kritische \emph{und} affirmative Theorie:
    \item[2.1] Die \emph{Kritik der zweiten Natur} als Selbstverfehlung des Geistes:
        Die ursprüngliche Freiheit der Setzung geht in der Wirklichkeit in den Freiheitsverlust über.
    \item[2.2] Die \emph{Affirmation der zweiten Natur} als Selbstverwirklichung des Geistes:
        Die Freiheit des Geistes realisiert sich gerade \qq{in der Doppelten Hervorbringung der Natur als seiner Welt und seiner Welt als Natur} (\cite[][145]{menke_autonomie_2018})
        Das bedeutet, dass der Geist Freiheit nur in dem Anderen gewinnt, was ihn einerseits ausdrückt, sich ihm gegenüber andererseits verselbstständigt (Bei-Sich-Selbst-Sein-im-Anderen). 
        Dieses Andere ist von ihm gesetzt (absoluter Geist) und zugleich nicht gesetzt (endlicher Geist).
    \item[2.3] Beides meint also das gleiche: \qq{die Selbsthervorbringung des Geistes als einer selbstständigen Natur} (\cite[][146]{menke_autonomie_2018}), nur einerseits als die Vollenung, einerseits, als der Fall des Geistes.
    \item[3.] Diese Einheit von Verfehlung und Gelingen, von Kritik und Affirmation ist nur so zu verstehen, dass beide Seiten untrennbar miteinander verbunden sind \emph{und} strikt voneinander getrennt.
    \item[3.1] Sie sind \emph{untrennbar verbunden}, weil der Geist in seiner Selbstverwirklichung seine Selbstverfehlung hervorbringt (absoluter Geist, schlägt mit der Verwirklichung um in endlichen Geist).
    Oder anders: \qq{Die Selbstverfehlung des Geistes setzt seine Selbstverwirklichung voraus} (\cite[][147]{menke_autonomie_2018}), aber ebenso die Selbstverwirklichung die Selbstverfehlung! \todo[noline]{Jonas Fragen}
    \item[3.2] Sie sind \emph{strikt getrennt}, weil die beiden Seiten der Verselbstständigung sich gegenseitig bekämpfen und verdrängen: \qq{das plötzliche hervortreten des Werkes, des Gedankens, der Tat bekämpft die immer gleiche Wiederholung abstrakt imaginärer Identitäten} (\cite[][147]{menke_autonomie_2018}) 
        \todo[noline]{Inwiefern ist dieses plötzliche Hervortreten möglich}
    \item[3.3] Nochmal in kurz: Die Selbstverwirklichung ist zugleich auch die Selbstverfehlung, daher sind sie verbunden, zugleich durchbricht die Vollendung die mechanische Wiederholung, so sind sie getrennt.
    \item[4.] Die Affirmation ist nicht positiv. 
        Das sie positiv wäre, würde bedeuten, dass sie das Gegebene bejaht, also das was ist (bloß weil es ist). 
        Das Was allerdings ist, ist der geistige Mechanismus.
        Diesen geistigen Mechanismus \emph{zerstört} aber die Affirmation. 
        Die Affirmation ist als neue zweite Natur \qq{die Zerstörung der falschen zweiten Natur} (\cite[][148]{menke_autonomie_2018}).
        Daher ist die Affirmation nicht positiv, sondern negativ.
    \item[4.1] Geneaure Bestimmung: \qq{Die Affirmation gilt dem Augenblick, in dem sich das geiste Sein soeben bereits gegenüber dem geistigen Setzen verselbstständigt hat und noch nicht in die Logik mechanischer Wiederholung eingetreten ist. [...] [Dadurch] ist sie die Rettung der zweiten Natur} (\cite[][148]{menke_autonomie_2018}).
\end{itemize}

\end{document}

