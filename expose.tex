\documentclass[a4paper, 12pt]{article}

\usepackage{secdot} % Dots in Section Numbers
\usepackage[utf8]{inputenc}
\usepackage[T1]{fontenc}
\usepackage[ngerman]{babel}
\usepackage[european]{circuitikz}
\usepackage{float}

\usepackage{fancyhdr}

\usepackage{todonotes}
\usepackage{hyperref}

\pagestyle{fancy}
\fancyhf{}
\lhead{Jan van Dick}
\chead{Exposè: \glqq Befreiung\grqq}
\rhead{\thepage}

\title{Exposé: \glqq Zweite Natur und Befreiung - zwischen Affirmation und Kritik\grqq}
\author{Jan van Dick}
\date{\today}

\begin{document}

\maketitle

\section{Abstrakt}
Die These vom Tode Gottes, dass \glqq wir\grqq{} ihn getötet haben, ist mehr als bloße Negativität. 
 Mord an Gott erhebt den Menschen selbst zum Gott. 
Der Mensch erkennt Gott als von ihm Gesetzt, als Schein, der sich gegen ihn verselbstständigt. 
Indem er dies erkennt und Gott tötet, den (von ihm gesetzen) Schein durchstößt, erkennt er seine eigene Fähigkeit, Gott geschaffen zu haben, also Sein zu Setzen.
Zugleich ist der Tod Gottes für die \glqq Rätselrather\grqq{} ein \glqq neues offenes Meer\grqq{}.
Ist das Erkennen dieser Fähigkeit des Geistes das unbekannte, neue offene, noch nie so offen gewesene Meer, oder ist es der Beginn des Wieder-in-Natur-Verkehrt-Seins, wie Menke es in der Analyse Hegels Begriffs der zweiten Natur.\\
In der folgenden Arbeit möchte ich das Thema zweite Natur und Befreiung aus Sicht Nietzsches und der von Christoph Menke rekonstruierten Perspektive Hegels untersuchen.
Dabei gilt es Zweite Natur als Kritik und Affirmation zu lesen. 
Sie bedeutet das in Natur verkehrt sein des Geistes, als auch die Kraft des Geistes Sein zu setzen. 
Während bei Hegel Befreiung aus der zweite Natur auf Grund der Endlichkeit des menschlichen Geistes wieder in Natur verfällt, scheint bei Nietzsche in der Metapher des neuen, offenen Meeres die Perspektive die Endlichkeit des menschlichen Geistes zu überwinden gegeben zu sein. 
Ob Nietzsche damit eine Ergänzung zu Hegels Philosophie darstellt bildet Gegenstand dieser Arbeit. 
Außerdem gilt es zu untersuchen, was Befreiung bei Hegel meint und wie Befreiung zu verstehen ist:
so ist Befreiung als Doppelschritt zuverstehen: 1. die Befreiung aus der ersten, äußeren Natur, die die zweite Natur hervorbring und 2. die Befreiung aus dieser vom Geist gesetzten zweiten Natur.
Diese Struktur ist so auch bei Nietzsche aufzufinden. 
Der Erste Teil der Arbeit wird demnach die Rekonstruktion des Hegelschen Begriffs der zweiten Natur darstellen. 
In einem zweiten Schritt soll dieser Begriff aus eine Nietzschianischen Perspektive Beschrieben werden. 
Während die Befreiung aus der 1. Natur unproblematische erschient, bleibt die Struktur Befreiung aus der zweiten Natur schleierhaft.
Diese Unklarheit soll in der Synthese von Hegels und Nietszches Position geklärt werden. 
Abschließend gilt es die These der notwendigen Wiederholung der zweiten Natur (auch nach der Befreiung) zu prüfen. 
In Nietzsches Werk scheint es eine Perpspektive der Überwindung der Notwendig-in-Natur-Verkehrtheit zu geben. 
Zu gleich betonen sowohl Hegel, als auch Nietzsche Freiheit als nicht-gegeben. 
Der Mensch ist nicht als solcher frei, sondern muss sich seine Freiheit immer wieder erkämpfen.
Freiheit ist die Befreiung aus der jeweiligen Unfreiheit (Adorno).\\
Die dialektik zwischen Freiheit und Notwendigkeit, Endlichkeit und Unendlichkeit des Geistes ist demnach Grundlage meiner Arbeit.
Die Leitfrage der Arbeit also, ob es in Nietszches Philosophie eine Möglichkeit der Überwindung Notwendigkeit-in-Natur-Verkehrtheit gibt, und ob diese Position haltbar ist.\\
Die Position Hegels, wird anhand von Menkes \glqq Autonomie und Befreiung\grqq, Hegels \glqq Grundlinien der Philosophie des Rechts\grqq und der \glqq Phänemenologie des Geites\grqq{} rekonstruiert.
Nietzsches Position soll an ausgewählen Abschnitten aus \glqq Die fröhliche Wissenschaft\grqq{} dargelegt werden.

\section{Vorläufiges Inhaltsverzeichnis}

\section{Literatur}
 
\end{document}
