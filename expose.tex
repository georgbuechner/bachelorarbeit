\documentclass[a4paper, 12pt]{article}

\usepackage{secdot} % Dots in Section Numbers
\usepackage[utf8]{inputenc}
\usepackage[ngerman]{babel}

\usepackage{fancyhdr}

\usepackage{todonotes}
\usepackage{hyperref}

\usepackage[autostyle]{csquotes}
\usepackage[
    backend=biber,
    style=alphabetic,
    sortlocale=de_DE,
    natbib=true,
    url=false,
    doi=true,
    eprint=false
    ]{biblatex}
\addbibresource{literatur.bib}

\pagestyle{fancy}
\fancyhf{}
\lhead{Jan van Dick}
\chead{Exposè: \glqq Zweite Natur und Befreiung\grqq}
\rhead{\thepage}

\title{Exposé: \glqq Zweite Natur und Befreiung - zwischen Affirmation und Kritik\grqq}
\author{Jan van Dick}
\date{\today}

\begin{document}

\maketitle

\section{Abstrakt}
Friedrich Nietzsches These vom Tode Gottes, dass \glqq wir\grqq{} ihn getötet haben, ist mehr als bloße Negativität. 
Der Mord an Gott erhebt den Menschen selbst zum Gott, nur so kann er ihn töten.\footfullcite[Vgl.][481]{nietzsche_morgenrote_1999}
Aber darin, dass er zum Gott werden muss, um ihn zu töten, liegt ebenso, dass er Gott gewesen sein muss, um ihn geschaffen zu haben.
Der Mensch erkennt Gott also als von ihm gesetzt, als Schein, der sich gegen ihn verselbstständigte.
Indem er dies erkennt und Gott tötet, den (von ihm gesetzen) Schein durchstößt, erkennt er seine eigene Fähigkeit, Gott geschaffen zu haben, also Sein zu Setzen. 
Das Durchstoßen des Scheins bringt die Umwälzung der gesamten europäischen Moral, den Zerfall der Sittlichkeit mit sich.
Zugleich ist der Tod Gottes für uns \glqq geborene Rätselrather\grqq{} ein \glqq neues offenes Meer\grqq .\footnote{\citeauthor[Vgl.][573]{nietzsche_morgenrote_1999}}
Ist nun Werden des Menschen zum Gott und das Erkennen seiner Kraft Sein zu Setzen das unbekannte, neue, offene, noch nie so offen gewesene Meer, oder ist es der Beginn des Wieder-in-Natur-Verkehrt-Seins, wie Menke es in der Analyse Hegels Begriffs der zweiten Natur beschreibt?\footfullcite[Vgl.][144]{menke_autonomie_2018}\\

In der folgenden Arbeit möchte ich das Thema zweite Natur und Befreiung aus Sicht Nietzsches und der von Christoph Menke rekonstruierten Perspektive Hegels untersuchen.
Dabei gilt es zweite Natur als Kritik und Affirmation zu lesen. 
Zweite Natur bedeutet, so verstanden, die notwenige In-Natur-Verkehrtheit des Geistes, als auch die Kraft des Geistes Sein zu setzen. 
In der zweiten Natur fallen Setzen und Sein ins Eins\footnote{\citeauthor[Vgl.][144]{menke_autonomie_2018}}.
Während bei Hegel Befreiung aus der zweiten Natur auf Grund der Endlichkeit des menschlichen Geistes wieder in Natur verfällt, scheint bei Nietzsche in der Metapher des neuen, offenen Meeres, die Perspektive die Endlichkeit des menschlichen Geistes zu überwinden, gegeben zu sein. 
Zu gleich betonen sowohl Hegel, als auch Nietzsche Freiheit als nicht-gegeben: 
Der Mensch ist nicht als solcher frei, sondern muss sich seine Freiheit immer wieder erkämpfen.
Freiheit ist die Befreiung aus der jeweiligen Unfreiheit\footfullcite[Vgl. Adorno][227]{adorno_negative_dialektik_2003}\\
Die Dialektik zwischen Freiheit und Notwendigkeit, Geist und Mechanismus, Endlichkeit und Unendlichkeit des Geistes, ist demnach Grundlage meiner Arbeit.
Die Leitfrage der Arbeit also: \textit{gibt es in Nietszches Philosophie eine Möglichkeit der Überwindung der Notwendigkeit-in-Natur-Verkehrtheit des Geistes gibt, und ist diese Position haltbar?}\\
Und ist diese Befreiung, Bewusstseins-Werden, eine Arbeit des Indiviuums, oder kann dieses Bewusstsein aus der Sittlichkeit selbst heraus hervorgebracht werden.
Verhältnis von Einzelheit und Sozialität.

\section{Struktur der Arbeit}
Befreiung ist bei Hegel als Doppelschritt zu verstehen: 1. die Befreiung aus der ersten, äußeren Natur, die die zweite Natur hervorbringt und 2. die Befreiung aus dieser, vom Geist gesetzten, zweiten Natur.
Diese Struktur ist so auch bei Nietzsche aufzufinden.\\
Der Erste Teil meiner Arbeit wird demnach die Rekonstruktion des Hegelschen Begriffs der zweiten Natur darstellen. 
Diese Rekonstruktion soll von folgendem Zitat Menkes ausgehen: 
\begin{itemize}
    \item[] Die Gewohnheit als zweite Natur [ist] geistig oder frei [...], insofern sie ein Ausdruck des Wollens (oder ein Setzen) ist, und [...] mechanisch oder unfrei [...], weil sie, einmal gesetzt, selbstständig und unbewußt wirkend ist.\footnote{\citeauthor[][145]{menke_autonomie_2018}}
\end{itemize}
Den Begriff der Gewohnheit, soll zunächst anhand der Erläuterung der Sittlichkeit als soziale Teilhabe beschrieben werden. Daraus ergibt sich der Begriff des \glqq geistigen Mechanismus\grqq\footnote{\citeauthor[][S.145]{menke_autonomie_2018}}. Die Rekonstruktion Hegels Position soll abschließend in der Dialektik von endlichem und unendlichem Geist erläutert werden.\\
In einem zweiten Schritt soll dieser Begriff aus einer nietzschianischen Perspektive beschrieben werden. 
Hier soll ausgehend von Nietzsches Begriff des Scheins, über den Unterschied von Rolle und Schauspieler, abschließend die These des Todes Gottes erläutert werden.\\
Während die Befreiung aus der 1. Natur unproblematisch erschient, bleibt die Struktur der Befreiung aus der zweiten Natur schleierhaft.
Diese Unklarheit gilt es in der Synthese von Hegels und Nietszches Position in einem dritten Schritt aufzudecken.\\
Abschließend gilt es die These der notwendigen Wiederholung der zweiten Natur (auch nach der Befreiung) zu prüfen.\\

Die Position Hegels wird anhand von Menkes \textit{Autonomie und Befreiung}, Hegels \textit{Grundlinien der Philosophie des Rechts} und der \textit{Phänemenologie des Geites} rekonstruiert.
Nietzsches Position soll an ausgewählen Abschnitten aus \textit{Die fröhliche Wissenschaft} dargelegt werden.


\section{Vorläufiges Inhaltsverzeichnis}
\begin{itemize}
    \item[I] Einleitung

    \item[II] Hauptteil
    \begin{enumerate}
        \item Zweite Natur bei Hegel
            \begin{itemize}
                \item Die Sittlichkeit
                \item Dialektik von Geist und Mechanismus
                \item Dialektik von Kritik und Affirmation
            \end{itemize}
        \item Zweite Natur bei Nietzsche
            \begin{itemize}
                \item Was bedeutet Schein?
                \item Der Schauspieler und die Rolle
                \item Gott ist Tot!
            \end{itemize}
        \item Was heißt Befreiung aus der zweiten Natur
        \item Das offene Meer?
    \end{enumerate}
    
    \item[III] Fazit
\end{itemize}


    
\printbibliography
 
\end{document}
